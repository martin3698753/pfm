\documentclass{article}
\usepackage[utf8]{inputenc}
\usepackage{blindtext}
\usepackage{graphicx}
\usepackage{amsmath}
\usepackage{csvsimple}
\usepackage{pdfpages}
\usepackage{hyperref}
\usepackage{caption}
\usepackage{subcaption}

\title{Objem D-rozměrné koule}
\author{Martin Skok}

\begin{document}
\maketitle

\begin{abstract}
  Střílení do koule o D rozměrech
\end{abstract}

\section{Zadání}
Úloha 5.4: Objem D-rozměrné koule
Určete metodou náhodného střílení objem koule v D-rozměrném prostoru, která je určena podmínkou
$$x_{1}^{2} + x_{2}^{2} \cdots + x_{D}^{2} \leq 1$$
kde
$$\vec{r} = (x_{1}, x_{2}, \cdots x_{D})$$
je polohový vektor v D rozměrném prostoru
Určete pro D = 2, 3 a alespoň dvě větší hodnoty
\section{Popis kódu}
Nejdříve jsem nadefinoval funkci \textbf{rnd}, která bude podle nastaveného poloměru koule
generovat náhodná čísla. Funkce \textbf{comp vol} potom podle definované dimenze v tomto rozmezí
nastřílí $N$ náhodných bodů a určí, jestli leží v kouli nebo ne.
Objem počítám jako
$$\frac{n_{circle}\cdot 2^{DIM}}{N}$$
Kde $n_{circle}$ je počet bodů v kouli, $DIM$ je dimenze prostoru a $N$ je celkový
počet výstřelů.\\
Tento proces opakuji $n$ krát a všechny hodnoty ukládám do $n$ velkého řetezce.
Poté ještě počítám průměr jako
$$\overline{V} = \frac{ \Sigma_{i}^{n}V_{i} }{n}$$
a střední kvadratickou odchylku
$$\sigma = \sqrt{\frac{ \Sigma_{i}^{n}(V_{i} - \overline{V})^{2} } {n(n-1)}}$$
Kde $V_{i}$ je objem z jednoho měření a $n$ je počet měření.
\newpage
Teoretickou hodnotu D-prostorové koule jsem počítal jako
$$V_{n}(R) = \frac{\pi^{n/2}}{\Gamma (\frac{n}{2} + 1)}R^{n}$$
kde $n$ je dimenze a $R$ je poloměr koule.

\section{Data}
Všechno jsem dělal jako jednotkovou kouli, tedy poloměr koule byl vždy jedna.
Počet nástřelů jsem zvolil $N = 1000000$ a počet opakování měření $n = 100$.
Toto mi přišlo jako ideální, jelikož mi vycházela malá chyba.
Nejdříve jsem spustil program 2 dimenze a 3 dimenze. Potom jsem udělal ještě
4 a 6 dimenzí.
  \subsection{2 DIM}
Teoretická hodnota pro tuto konfiguraci byla $V_{ter} = 3.141593 \approx \pi$\\
Nasimulovaný průměr mi vyšel $\overline{V} = 3.141618$\\
a odchylka mi vyšla $\sigma = 0.00016 = 0.016 \%$
\subsection{3 DIM}
$V_{ter} = 4.188790$\\
$\overline{V} = 4.188328$\\
$\sigma = 0.000383 = 0.0383 \%$
\subsection{4 DIM}
$V_{ter} = 4.934802$\\
$\overline{V} = 4.934809$\\
$\sigma = 0.000825 = 0.0825 \%$
\subsection{6 DIM}
$V_{ter} = 5.167713$\\
$\overline{V} = 5.165654$\\
$\sigma = 0.001668 = 0.1668 \%$
\newpage
\section{Závěr}
Podařilo se mi úspěšně určit objem 2, 3, 4 a 6 dimenzionálních koulí. Nasimulované hodnoty
se blíží k teoretickým hodnotám.
\end{document}
